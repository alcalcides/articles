% Copyright 2004 by Till Tantau <tantau@users.sourceforge.net>.
%
% In principle, this file can be redistributed and/or modified under
% the terms of the GNU Public License, version 2.
%
% However, this file is supposed to be a template to be modified
% for your own needs. For this reason, if you use this file as a
% template and not specifically distribute it as part of a another
% package/program, I grant the extra permission to freely copy and
% modify this file as you see fit and even to delete this copyright
% notice. 

\documentclass{beamer}

% There are many different themes available for Beamer. A comprehensive
% list with examples is given here:
% http://deic.uab.es/~iblanes/beamer_gallery/index_by_theme.html
% You can uncomment the themes below if you would like to use a different
% one:
%\usetheme{AnnArbor}
%\usetheme{Antibes}
%\usetheme{Bergen}
%\usetheme{Berkeley}
%\usetheme{Berlin}
%\usetheme{Boadilla}
%\usetheme{boxes}
%\usetheme{CambridgeUS}
%\usetheme{Copenhagen}
%\usetheme{Darmstadt}
%\usetheme{default}
%\usetheme{Frankfurt}
%\usetheme{Goettingen}
%\usetheme{Hannover}
%\usetheme{Ilmenau}
%\usetheme{JuanLesPins}
%\usetheme{Luebeck}
\usetheme{Madrid}
%\usetheme{Malmoe}
%\usetheme{Marburg}
%\usetheme{Montpellier}
%\usetheme{PaloAlto}
%\usetheme{Pittsburgh}
%\usetheme{Rochester}
%\usetheme{Singapore}
%\usetheme{Szeged}
%\usetheme{Warsaw}
\usepackage[brazil]{babel}
\usepackage[utf8]{inputenc}

\title{Sistemas Operacionais em Assembly}

% A subtitle is optional and this may be deleted
\subtitle{Uma abordagem "top-bottom" com foco prático}

\author{Alcides Augusto
  %\inst{1} 
  \and 
  Paulo Raimundo
  %\inst{2}
}
% - Give the names in the same order as the appear in the paper.
% - Use the \inst{?} command only if the authors have different
%   affiliation.

\institute[Universidade Federal da Bahia] % (optional, but mostly needed)
{
  %\inst{1}%
  Colegiado de Engenharia da Computação\\
  Universidade Federal da Bahia\\
  %\and
  %\inst{2}%
  %Department of Theoretical Philosophy\\
  %University of Elsewhere
}
% - Use the \inst command only if there are several affiliations.
% - Keep it simple, no one is interested in your street address.

\date{MATA49 - Prog Software Básico, 2019-1}
% - Either use conference name or its abbreviation.
% - Not really informative to the audience, more for people (including
%   yourself) who are reading the slides online

%\subject{Programação de Software Básico}
% This is only inserted into the PDF information catalog. Can be left
% out. 

% If you have a file called "university-logo-filename.xxx", where xxx
% is a graphic format that can be processed by latex or pdflatex,
% resp., then you can add a logo as follows:

% \pgfdeclareimage[height=0.5cm]{university-logo}{university-logo-filename}
% \logo{\pgfuseimage{university-logo}}

% Delete this, if you do not want the table of contents to pop up at
% the beginning of each subsection:
\AtBeginSubsection[]
{
  \begin{frame}<beamer>{Outline}
    \tableofcontents[currentsection,currentsubsection]
  \end{frame}
}

% Let's get started
\begin{document}

\begin{frame}
  \titlepage
\end{frame}

\begin{frame}{Roteiro}
  \tableofcontents
  % You might wish to add the option [pausesections]
\end{frame}

% Section and subsections will appear in the presentation overview
% and table of contents.
\section{Introdução}
\subsection{Hardware}
\subsection{Software}
\subsection{Hard vs Software}

\section{SOs: fundamentos}

\subsection{Objetivos de um SO}
\subsection{Elementos de um SO}
\subsection{Níveis de privilégio}
\subsection{Chamadas de sistema}

\section{SO na prática}

\subsection{Suse Studio}
\subsection{MikeOS}
\subsection{MenuetOS}

\section{Introdução}

\subsection{Hardware}

\begin{frame}{Introdução}
  O que é um Sistema Computacional?
\end{frame}

\begin{frame}{Introdução}
  O que é um Sistema Computacional?
  \begin{itemize}
    \item Hardware
    \item Software
  \end{itemize}
\end{frame}


\begin{frame}{Introdução}{Hardware}
  O que é o Hardware?
\end{frame}

\begin{frame}{Introdução}{Hardware}
  Tudo bem\dots Todo mundo aqui já sabe\dots Dispensa comentários\dots
\end{frame}

\begin{frame}{Introdução}{Hardware}
  Mas permita-nos apenas destacar:
  \begin{itemize}
    \item O Hardware é máquina física\dots
  \end{itemize}
\end{frame}

\begin{frame}{Introdução}{Hardware}
  Permita-nos apenas destacar:
  \begin{itemize}
    \item O Hardware é máquina física que serve para\dots   
  \end{itemize}
\end{frame}

\begin{frame}{Introdução}{Hardware}
  Permita-nos apenas destacar:
  \begin{itemize}
    \item O Hardware é máquina física que serve para 
    \item Fazer OPERAÇÕES ALGORÍTMICAS
  \end{itemize}
\end{frame}

\begin{frame}{Introdução}{Hardware}
  Permita-nos apenas destacar:
  \begin{itemize}
    \item O Hardware é máquina física que serve para 
    \item Fazer OPERAÇÕES ALGORÍTMICAS
    \item Mas para que fazer tantas OPERAÇÕES ALGORÍTMICAS???
  \end{itemize}
\end{frame}


\subsection{Software}

\begin{frame}{Introdução}{Software}
  \begin{itemize}
    \item Funcionalidades que o computador é capaz de promover
    \end{itemize}
\end{frame}

\begin{frame}{Introdução}{Software}
  \begin{itemize}
    \item Funcionalidades que o computador é capaz de promover
    \item Através de seus programas
    \end{itemize}
\end{frame}

\begin{frame}{Introdução}{Software}
  \begin{itemize}
    \item Funcionalidades que o computador é capaz de promover
    \item Através de seus programas
    \item Essas funcionalidades são a abstração de programas
    \end{itemize}
\end{frame}

\begin{frame}{Introdução}{Software}
  \begin{itemize}
    \item Funcionalidades que o computador é capaz de promover
    \item Através de seus programas
    \item Essas funcionalidades são a abstração de programas
      \begin{itemize}
        \item são o conjunto de instruções algorítmicas
      \end{itemize}
    \end{itemize}
\end{frame}

\begin{frame}{Introdução}{Software}
  \begin{itemize}
    \item Funcionalidades que o computador é capaz de promover
    \item Através de seus programas
    \item Essas funcionalidades são a abstração de programas
      \begin{itemize}
        \item são o conjunto de instruções algorítmicas 
        \item são geradas através das linguagens de programação.
      \end{itemize}
    \end{itemize}
\end{frame}

\subsection{Hard vs Software}

\begin{frame}{Introdução}{Hard vs Software}
  \begin{itemize}
    \item O relacionamento entre Hardware e Software é "sensível"\dots
      \begin{itemize}
        \item Correspondência do tipo um-para-um?
          \begin{itemize}
            \item instruções da linguagem de programação 
            \item instruções binárias do hardware
          \end{itemize}
        \item Tradução adequada de instruções
      \end{itemize}
  \end{itemize}
\end{frame}

\begin{frame}{Introdução}{Hard vs Software}
  \begin{itemize}
    \item O relacionamento entre Hardware e Software é "sensível"\dots
      \begin{itemize}
        \item Há vários tipos de processadores!
      \end{itemize}
  \end{itemize}
\end{frame}


\begin{frame}{Introdução}{Hard vs Software}
  \begin{itemize}
    \item O relacionamento entre Hardware e Software é "sensível"\dots
      \begin{itemize}
        \item Há vários tipos de processadores!
        \item Portabilidade?!
      \end{itemize}
  \end{itemize}
\end{frame}

\begin{frame}{Introdução}{Hard vs Software}
  \begin{itemize}
    \item O relacionamento entre Hardware e Software é "sensível"\dots
      \begin{itemize}
        \item Há vários tipos de processadores!
        \item Portabilidade?!
        \item Segurança de dados?!
      \end{itemize}
  \end{itemize}
\end{frame}

\begin{frame}{Introdução}{Hard vs Software}
  \begin{itemize}
    \item O relacionamento entre Hardware e Software é "sensível"\dots
      \begin{itemize}
        \item Há vários tipos de processadores!
        \item Portabilidade?!
        \item Segurança de dados?!
        \item Gestão de recursos computacionais?!
      \end{itemize}
  \end{itemize}
\end{frame}

\begin{frame}{Introdução}{Hard vs Software}
  E agora o que fazer?
\end{frame}

\begin{frame}{Introdução}{Hard vs Software}
  Crie um \emph{Sistema Operacional!}
\end{frame}

\begin{frame}{Introdução}{Hard vs Software}
  Crie um \emph{Sistema Operacional!}\\
  "Solução do lado do Software"
\end{frame}

\section{SOs: fundamentos}
\subsection{Objetivos de um SO}

\begin{frame}{SOs: fundamentos}{Objetivos de um SO}
  \begin{itemize}
  \item Abstração
  \item Gerência
  \end{itemize}
\end{frame}

\subsection{Elementos de um SO}

\begin{frame}{SOs: fundamentos}{Elementos de um SO}
  \begin{itemize}
  \item Programas utilitários
  \item Drivers
  \item Código de inicialização (boot code)
  \item Núcleo (kernel)
  \end{itemize}
\end{frame}

\begin{frame}{First Slide Title}{Optional Subtitle}
  \begin{itemize}
  \item {
    My first point.
  }
  \item {
    My second point.
  }
  \end{itemize}
\end{frame}

\begin{frame}{First Slide Title}{Optional Subtitle}
  \begin{itemize}
  \item {
    My first point.
  }
  \item {
    My second point.
  }
  \end{itemize}
\end{frame}

\subsection{Second Subsection}

% You can reveal the parts of a slide one at a time
% with the \pause command:
\begin{frame}{Second Slide Title}
  \begin{itemize}
  \item {
    First item.
    \pause % The slide will pause after showing the first item
  }
  \item {   
    Second item.
  }
  % You can also specify when the content should appear
  % by using <n->:
  \item<3-> {
    Third item.
  }
  \item<4-> {
    Fourth item.
  }
  % or you can use the \uncover command to reveal general
  % content (not just \items):
  \item<5-> {
    Fifth item. \uncover<6->{Extra text in the fifth item.}
  }
  \end{itemize}
\end{frame}

\section{Second Main Section}

\subsection{Another Subsection}

\begin{frame}{Blocks}
\begin{block}{Block Title}
You can also highlight sections of your presentation in a block, with it's own title
\end{block}
\begin{theorem}
There are separate environments for theorems, examples, definitions and proofs.
\end{theorem}
\begin{example}
Here is an example of an example block.
\end{example}
\end{frame}

% Placing a * after \section means it will not show in the
% outline or table of contents.
\section*{Summary}

\begin{frame}{Summary}
  \begin{itemize}
  \item
    The \alert{first main message} of your talk in one or two lines.
  \item
    The \alert{second main message} of your talk in one or two lines.
  \item
    Perhaps a \alert{third message}, but not more than that.
  \end{itemize}
  
  \begin{itemize}
  \item
    Outlook
    \begin{itemize}
    \item
      Something you haven't solved.
    \item
      Something else you haven't solved.
    \end{itemize}
  \end{itemize}
\end{frame}



% All of the following is optional and typically not needed. 
\appendix
\section<presentation>*{\appendixname}
\subsection<presentation>*{For Further Reading}

\begin{frame}[allowframebreaks]
  \frametitle<presentation>{For Further Reading}
    
  \begin{thebibliography}{10}
    
  \beamertemplatebookbibitems
  % Start with overview books.

  \bibitem{Author1990}
    A.~Author.
    \newblock {\em Handbook of Everything}.
    \newblock Some Press, 1990.
 
    
  \beamertemplatearticlebibitems
  % Followed by interesting articles. Keep the list short. 

  \bibitem{Someone2000}
    S.~Someone.
    \newblock On this and that.
    \newblock {\em Journal of This and That}, 2(1):50--100,
    2000.
  \end{thebibliography}
\end{frame}

\end{document}


